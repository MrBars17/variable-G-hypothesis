\begin{abstract}
We propose a phenomenological time--dependent Newton constant $G(t)$
that starts from zero at the Big Bang, quickly rises to $0.85\,G_0$
during the first 200 s, grows logistically to $0.98\,G_0$ by
recombination (380 kyr), and asymptotically reaches $G_0$ today.
This single trajectory simultaneously (i) lowers the primordial
$^7$Li abundance by $\sim30\%$, (ii) increases the CMB--inferred
Hubble constant by $\sim5\%$, and (iii) suppresses the growth of
matter fluctuations, alleviating the $\sigma_8$ tension.  A future
$+2\%$ drift in $G$ could lead to a “Big Hole” horizon‑percolation
scenario.  We list concrete observational tests: next‑generation
LLR ($|\dot G/G|\!\approx\!10^{-14}\,\mathrm{yr^{-1}}$), CMB‑S4
($\Delta r_s/r_s\!\approx\!0.5\%$) and Euclid/SKA constraints on
$\sigma_8(z)$.
\end{abstract}